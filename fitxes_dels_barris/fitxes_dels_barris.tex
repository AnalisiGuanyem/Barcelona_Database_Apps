\documentclass[]{article}
\usepackage[T1]{fontenc}
\usepackage{lmodern}
\usepackage{amssymb,amsmath}
\usepackage{ifxetex,ifluatex}
\usepackage{fixltx2e} % provides \textsubscript
% Set line spacing
% use upquote if available, for straight quotes in verbatim environments
\IfFileExists{upquote.sty}{\usepackage{upquote}}{}
\ifnum 0\ifxetex 1\fi\ifluatex 1\fi=0 % if pdftex
  \usepackage[utf8]{inputenc}
\else % if luatex or xelatex
  \ifxetex
    \usepackage{mathspec}
    \usepackage{xltxtra,xunicode}
  \else
    \usepackage{fontspec}
  \fi
  \defaultfontfeatures{Mapping=tex-text,Scale=MatchLowercase}
  \newcommand{\euro}{€}
\fi
% use microtype if available
\IfFileExists{microtype.sty}{\usepackage{microtype}}{}
\usepackage[margin=1in]{geometry}
\usepackage{longtable,booktabs}
\usepackage{graphicx}
% Redefine \includegraphics so that, unless explicit options are
% given, the image width will not exceed the width of the page.
% Images get their normal width if they fit onto the page, but
% are scaled down if they would overflow the margins.
\makeatletter
\def\ScaleIfNeeded{%
  \ifdim\Gin@nat@width>\linewidth
    \linewidth
  \else
    \Gin@nat@width
  \fi
}
\makeatother
\let\Oldincludegraphics\includegraphics
{%
 \catcode`\@=11\relax%
 \gdef\includegraphics{\@ifnextchar[{\Oldincludegraphics}{\Oldincludegraphics[width=\ScaleIfNeeded]}}%
}%
\ifxetex
  \usepackage[setpagesize=false, % page size defined by xetex
              unicode=false, % unicode breaks when used with xetex
              xetex]{hyperref}
\else
  \usepackage[unicode=true]{hyperref}
\fi
\hypersetup{breaklinks=true,
            bookmarks=true,
            pdfauthor={Grup d'Anàlisi de Dades},
            pdftitle={Perfil de barri o districte},
            colorlinks=true,
            citecolor=blue,
            urlcolor=blue,
            linkcolor=magenta,
            pdfborder={0 0 0}}
\urlstyle{same}  % don't use monospace font for urls
\setlength{\parindent}{0pt}
\setlength{\parskip}{6pt plus 2pt minus 1pt}
\setlength{\emergencystretch}{3em}  % prevent overfull lines
\setcounter{secnumdepth}{0}

%%% Change title format to be more compact
\usepackage{titling}
\setlength{\droptitle}{-2em}
  \title{Perfil de barri o districte}
  \pretitle{\vspace{\droptitle}\centering\huge}
  \posttitle{\par}
  \author{Grup d'Anàlisi de Dades}
  \preauthor{\centering\large\emph}
  \postauthor{\par}
  \predate{\centering\large\emph}
  \postdate{\par}
  \date{31/01/2015}




\begin{document}

\maketitle


\section{DISTRICTE de `SANT MARTÍ'}\label{districte-de-sant-marti}

\subsection{Dades polítiques}\label{dades-politiques}

\begin{longtable}[c]{@{}lll@{}}
\toprule\addlinespace
\textbf{\ldots{}} & \textbf{al districte} & \textbf{a la ciutat}
\\\addlinespace
\midrule\endhead
\\\addlinespace
Abstenció eleccions generals 2011 & 32.3 & 31.9
\\\addlinespace
\\\addlinespace
Abstenció eleccions locals 2011 & 48 & 47
\\\addlinespace
Vot a CIU eleccions locals 2011 & 21.7 & 28.2
\\\addlinespace
Vot a PSC eleccions locals 2011 & 25.7 & 21.8
\\\addlinespace
Vot a PP eleccions locals 2011 & 16.6 & 16.9
\\\addlinespace
Vot a ICV eleccions locals 2011 & 11.5 & 10.2
\\\addlinespace
Vot a ERC eleccions locals 2011 & 5.8 & 5.5
\\\addlinespace
\\\addlinespace
Vot a CIU eleccions europees 2014 & 14.9 & 20.7
\\\addlinespace
Vot a PSC eleccions europees 2014 & 14.4 & 12.1
\\\addlinespace
Vot a PP eleccions europees 2014 & 10.8 & 11.9
\\\addlinespace
Vot a ICV eleccions europees 2014 & 14.4 & 12.5
\\\addlinespace
Vot a ERC eleccions europees 2014 & 22.9 & 21.6
\\\addlinespace
Vot a Podemos eleccions europees 2014 & 5.6 & 4.7
\\\addlinespace
\\\addlinespace
\bottomrule
\end{longtable}

\subsection{Dades socio-econòmiques}\label{dades-socio-economiques}

\begin{longtable}[c]{@{}lll@{}}
\toprule\addlinespace
\textbf{\ldots{}} & \textbf{al districte} & \textbf{a la ciutat}
\\\addlinespace
\midrule\endhead
\\\addlinespace
Atur (estimació)\footnote{Disposem a nivell de barri de dades sobre la
  població i sobre el nombre d'aturats registrats. Per calcular la taxa
  d'atur cal saber la taxa d'activitat (no disponible per barris) i el
  nombre d'aturats real (més alt que els registrats). Per tant, cal fer
  una estimació aproximada. Usant com a referència dades de la EPA del
  4rt trimestre de 2013, hem multiplicat la població total per 0.51 per
  estimar la població activa i hem multiplicat l'atur registrat per 1.38
  per estimar la població aturada.} & 18.8 & 16.9
\\\addlinespace
\\\addlinespace
Renda familiar\footnote{Es tracta de la `renda familiar bruta
  disponible', un índex en base a 100 que calcula el servei
  d'estadística de l'Ajuntament.} 2014 & 81.2 & 100
\\\addlinespace
\\\addlinespace
\bottomrule
\end{longtable}

\subsection{Habitatge}\label{habitatge}

\begin{longtable}[c]{@{}lll@{}}
\toprule\addlinespace
\textbf{\ldots{}} & \textbf{al districte} & \textbf{a la ciutat}
\\\addlinespace
\midrule\endhead
\\\addlinespace
Preu metre quadrat compra 2013 & 2097 & 2454
\\\addlinespace
Preu metre quadrat lloguer 2011 & 11.7 & 11.9
\\\addlinespace
\\\addlinespace
Persones per llar & 2.5 & 2.4
\\\addlinespace
Densitat (habitants/hectàrea) & 221.3 & 157.8
\\\addlinespace
\\\addlinespace
Llars comprades i pagades (\%) & 41 & 37.6
\\\addlinespace
Llars de lloguer (\%) & 22 & 30.1
\\\addlinespace
Llars comprades hipotecades\footnote{El percentatge de llars amb
  diferent tipus de tinença no sumen 100 perquè hi ha altres tipus (ex.:
  `herència', `cedides', \ldots{}) no recollits en aquesta taula.} (\%)
& 28.5 & 22.5
\\\addlinespace
\\\addlinespace
Llars sense calefacció (\%) & 10.7 & 10.7
\\\addlinespace
Llars sense internet\footnote{Les dades sobre percentatges de llars amb
  manques (calefacció, internet) poden contenir errors perquè sovint hi
  ha sobre-estimació del nombre de llars (denominador), disminuint
  llavors el percentatge real.} (\%) & 34.8 & 34.6
\\\addlinespace
\\\addlinespace
\bottomrule
\end{longtable}

\subsection{Població}\label{poblacio}

\begin{longtable}[c]{@{}lll@{}}
\toprule\addlinespace
\textbf{\ldots{}} & \textbf{al districte} & \textbf{a la ciutat}
\\\addlinespace
\midrule\endhead
\\\addlinespace
Població a Barcelona el 2013 (en milers) & 233 & 1612
\\\addlinespace
Nascuts a Catalunya (\%) & 59.3 & 59.2
\\\addlinespace
Nascuts a Espanya fora de Catalunya (\%) & 24.7 & 23.3
\\\addlinespace
Nascuts a l'extranger\footnote{Els percentatges no sumen exactament 100
  per diferències entre les fonts. Són estimacions aproximades.} (\%) &
19.8 & 21.8
\\\addlinespace
\\\addlinespace
\bottomrule
\end{longtable}

\subsection{Dades socio-demogràfiques}\label{dades-socio-demografiques}

\begin{longtable}[c]{@{}lll@{}}
\toprule\addlinespace
\textbf{\ldots{}} & \textbf{al districte} & \textbf{a la ciutat}
\\\addlinespace
\midrule\endhead
\\\addlinespace
Taxa de natalitat 2014 & 9 & 8.2
\\\addlinespace
Taxa de mortalitat 2014 & 8.1 & 9.2
\\\addlinespace
Taxa bruta d'immigració 2014 & 42.9 & 47.1
\\\addlinespace
Taxa bruta d'emigració 2014\footnote{Taxes per cada 1000 habitants.} &
32.1 & 33.3
\\\addlinespace
\\\addlinespace
Índex de dependència demogràfica\footnote{Calculat pel servei
  d'Estadística de l'Ajuntament en base a la seguent fórmula: (Població
  65 i més / població de 16 a 64) x100} & 52.1 & 53
\\\addlinespace
Menors de 14 anys (\%) & 13.3 & 12.4
\\\addlinespace
Majors de 65 anys (\%) & 0 & 0
\\\addlinespace
\\\addlinespace
\bottomrule
\end{longtable}

\subsection{Nivell educatiu}\label{nivell-educatiu}

\begin{longtable}[c]{@{}lll@{}}
\toprule\addlinespace
\textbf{\ldots{}} & \textbf{al districte} & \textbf{a la ciutat}
\\\addlinespace
\midrule\endhead
\\\addlinespace
Estudis primaris (\%) & 25.2 & 22.6
\\\addlinespace
Estudis secundaris (\%) & 41.7 & 40.3
\\\addlinespace
Estudis universitaris (\%) & 19.3 & 23.9
\\\addlinespace
\\\addlinespace
\bottomrule
\end{longtable}

Nota: Moltes estadístiques oficials es calculen en base a enquestes que
no són representatives, sovint ni tant sols a nivell de districte. Per
exemple, la taxa d atur, que és el percentatge d'aturats sobre la
població activa, no es pot estimar a nivell de barri perquè la població
activa es calcula en base a la EPA, que només és representativa a nivell
de ciutat. En aquestes circunstàncies només es pot fer una aproximació
en base al nombre d'aturats i el nombre de persones en edat de
treballar. Un altre cas similar és la taxa de risc de pobresa, que es
calcula en base a l'enquesta de condicions de vida de tota la ciutat.

\newpage

\section{\textbf{Dades subjectives}}\label{dades-subjectives}

La informació que apareix a continuació a sigut recollida mitjançant
enquestes fetes a través d'internet
(\href{http://goo.gl/OfqwdC}{\url{http://goo.gl/OfqwdC}}). Tot i que
l'enquesta segueix en funcionamenta, les dades que aquí es presentan van
ser descarregades el 28 de gener del 2015. L'objectiu d'aquesta
descàrrega inicial és la de poder aportar informacions que puguin ser
d'utilitat pel diagnóstic dels barris. És molt important ser conscients
de que \textbf{la informació que presentem no té cap significació
estadística} (i menys a nivell de barris) \textbf{que permeti fer cap
tipus d'inferència}.

En pràcticament tots els casos, la mostra aconseguida no és prou gran. A
més, aquesta mostra no és mai, com veurem a continuació, aleatòria. És
per això que demanem NO pendre les informacions recollides en aquesta
secció com a dades representatives del conjunt de la població del barri
o districte, sinó més aviat com a opinions puntuals a títol personal.
Per descomptat, aquests apunts sobre el conjunt de la mostra no li
treuen gens d'importància a cadascuna de les valuoses opinions
recollides i que també expliquen part de la realitat del barri o
districte.

\subsection{Enquestes al districte de `SANT MARTÍ'
(mostra)}\label{enquestes-al-districte-de-sant-marti-mostra}

La mostra consta de 108 enquestes realitzades des de 72 IPs diferents.
Com es pot comprovar a continuació, el perfil de les persones
enquestades no és representatiu del conjunt de la població, mostrant
alguns biaixos importants:

.

\includegraphics{fitxes_dels_barris_files/figure-latex/INTERVIEW_PLOTS-1.pdf}

\subsection{Opinions sobre la situació al
barri}\label{opinions-sobre-la-situacio-al-barri}

\includegraphics{fitxes_dels_barris_files/figure-latex/HOUSING_OPINION_PLOTS-1.pdf}

Altres problemes del barri relacionats amb la vivenda i enumerats durant
les enquestes (sense filtratge previ): \emph{NA / atur i precarietat /
manca d'ajudes / maleït 22@ / cap / pisos turístics a preus desorbitats,
naus industrials sobrepoblades i en condicions infra-humanes\ldots{}s /
el problema al meu barri no el sé, però el que veig és que cada cop hi
ha més gent sense sostre!! / condiciones fiscales desfavorables. /
gentrificació, elevat preu, venta de pisos a estrangers adinerats / no
lo se / que el preu està molt per sobre del preu mde cost}.

\end{document}
